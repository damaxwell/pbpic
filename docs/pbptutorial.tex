\documentclass[12pt]{article}
\usepackage[left=1in,right=1in,top=1in,bottom=1in]{geometry}
\usepackage{pbpic,graphicx}
\usepackage{amsmath,amsfonts}
\usepackage[lf]{MinionPro}
\usepackage{fancyvrb}
\def\prg#1{{\tt #1}}
\def\pbp{Proof by Picture }
\def\Python{python}
\def\cmd#1{{\tt #1}}
\def\file#1{{\tt #1}}
\def\code#1{{\tt #1}}
\def\define#1{{\bf #1}}
\usepackage{listings}
\lstnewenvironment{snippet}{\lstset{language=Python,xleftmargin=12pt, xrightmargin=12pt,basicstyle=\footnotesize\ttfamily,escapechar=|}}{}

\begin{document}
\section{Introduction}
Proof by Picture is a package for making mathematical diagrams
using the Python programming language.  This document teaches how to use 
Proof by Picture by walking through a number of
increasingly sophisticated diagrams.  Before beginning
the tutorial you will want to have installed Proof by Picture according to
the installation guide, and verified that you can run Python to
create a diagrams with it.

\section{Book I Proposition 1}
The first proposition of Euclid's Elements states that 
given a line segment $AB$, one can
construct an equilateral triangle such that
its base is $AB$.

\hfil\begin{pbpic}<7*cm,5*cm>
scaleto(1*cm)
translate(loc.center)

R=1.5
A=point(-R/2,0)
B=point(R/2,0)
with ctmsave():
  translate(A)
  rotate(1./6)
  C=pagePoint(R,0)

setlinewidth(0.5*pt)
for c in [A,B]:
  moveto(c); build(paths.circle,r=R)
  stroke()

setlinewidth(1*pt)
moveto(A); lineto(B); lineto(C); closepath();
stroke()

moveto(A); rmoveto(-4*pt,0)
drawtex("$A$",origin=loc.east)

moveto(B); rmoveto(4*pt,0)
drawtex("$B$",origin=loc.west)
\end{pbpic}

The starting point for making this diagram in Proof by Picture is the
following code block.
\begin{snippet}
from pbpic import *

pbpbegin(7*cm,5*cm,'pdf')

pbpend()
\end{snippet}
These three commands constitute a minimal Proof by Picture diagram.
The \code{pbpbegin} command sets up the dimensions of the figure
(in this case it is 7cm wide and 5cm high) and the desired output file
and format.  We'll generate a PDF diagram, and the file name is
derived from the name of the Python file containing the diagram commands.
For example, if this minimal diagram was described in the file \file{euclid-I.1.py} then the output would be saved to \file{euclid-I.1.pdf}.
All of the drawing is done between 
\code{pbpbegin} and  the terminating \code{pbpend}
that writes the final diagram to the output file.

The default coordinate system for a diagram has its origin
in the lower-left corner and has its unit size selected
so that one unit is 1 point (i.e. $1/72$ of an inch).
Moving
in the positive $x$-direction corresponds to moving right, while moving
in the positive $y$-direction corresponds to moving up.  We set up
a more convenient coordinate system for our picture via
\begin{snippet}
scaleto(1*cm)
translate(loc.center)
\end{snippet}
The first command rescales the unit length so that one unit corresponds to 
1cm, and the second translates the origin of the coordinate system 
so that it lies at the center of the diagram.

\hfil\begin{pbpic}<8*cm,6*cm>
import pbpic.arrow
arrowto=arrow.ArrowTo();

W=7.;H=5.

scaleto(1*cm)
translate(0.5,0.5)

with gsave():
  moveto(0,0); build(paths.rect,w=W,h=H)
  clip()
  
  setlinewidth(2)
  setlinecolor(color.GrayColor(0.2))
  moveto(-0.5,-0.5); build(paths.vlines,h=H+1,M=int(W)+2)
  moveto(-0.5,-0.5); build(paths.hlines,w=W+1,N=int(H)+2)
  stroke()

moveto(0,0); build(paths.rect,w=7,h=5)
stroke()

translate(W/2,H/2)
moveto(0,0)
draw(arrowto,(1,0))
moveto(0,0)
draw(arrowto,(0,1))
\end{pbpic}

The circles in the diagram will have radius $R=1.5$cm, so the
points $A$ and $B$ will have coordinates $(-R/2,0)$ and $(R/2,0)$
in the new coordinate system. The location of the third vertex
of the triangle can be computed easily using a little trigonometry,
but it is easier to let the computer do the work.
\begin{snippet}
R=1.5
A=point(-R/2,0)
B=point(R/2,0)
with ctmsave():
  translate(A)
  rotate(1./6)
  C=pagePoint(R,0)
\end{snippet}
The `\code{with ctmsave():}' command saves the current coordinate system.
At the exit of the code block, the coordinate system is reset to whatever
it was when the \code{ctmsave} command was issued.  For the record, \code{ctm}
stands for `current transformation matrix'; we'll say more about transformation matrices
later.  In this example, we translate the origin so that it lies on the
point $A$ and then rotate through 1/6 of a circle. 

\hfil\begin{pbpic}<8*cm,6*cm>
import pbpic.arrow
arrowto=arrow.ArrowTo();

W=7.;H=5.

scaleto(1*cm)
translate(loc.center)

R=1.5
A=pagePoint(-R/2,0)
B=pagePoint(R/2,0)

with gsave():

  moveto(-W/2,-H/2); build(paths.rect,w=W,h=H)
  clip()
  
  translate(A)
  rotate(1./6)
  C=pagePoint(R,0)
  setlinewidth(2)
  setlinecolor(color.GrayColor(0.2))
  moveto(-5,-5); build(paths.vlines,h=10,M=11)
  moveto(-5,-5); build(paths.hlines,w=10,N=11)
  stroke()

moveto(-W/2,-H/2); build(paths.rect,w=W,h=H)
stroke()

with ctmsave():
  translate(A)
  scaleto(1*pt)
  moveto(A); build(paths.circle,r=2)
  moveto(C); build(paths.circle,r=2)  
  fill()

moveto(A); rmoveto(-0*pt,-4*pt);
drawtex('$A$',origin=loc.north)

moveto(C); rmoveto(4*pt,0*pt);
drawtex('$C$',origin=loc.west)


translate(A)
rotate(1./6)
moveto(0,0)
draw(arrowto,(1,0))
moveto(0,0)
draw(arrowto,(0,1))

\end{pbpic}

In the new coordinate system, $C$ lies at the point $(R,0)$.  We
save the location of this point using the command \code{C=pagePoint(R,0)}.
The variable \code{C} then 
contains a \code{PagePoint} object that represents a point 
on the page without reference to any particular coordinate system.

The next step is to draw the two circles.  Marks on the page
in Proof by Picture figure are described by a fundamental object known
as a path.  To draw a circle, we first build the path corresponding
to the circle, and then stroke it.
\begin{snippet}
setlinewidth(0.5*pt)
for c in [A,B]:
  moveto(c);
  build(paths.circle,r=R)
  stroke()
\end{snippet}
At any moment while drawing a Proof by Picture diagram, there is a notion
of the current path.  There are primitive path construction commands, e.g.
\code{moveto}, \code{lineto}, \code{curveto}, as well as a mechanism for
constructing more elaborate paths based on these primitives using the
\code{build} command.  
Associated with the current path is the notion of the \define{current point}, which is the last location mentioned in the current path. 
To build a circle, we start with a \code{moveto} command 
which effectively relocates the current point without 
adding anything of substance to the path.  Next, 
\code{build(paths.circle,r=R)} adds a circle to the current path.
The circle is centered at the current point and has 
the specified radius.  The \code{stroke} command draws the outline
of the current path to the page, and then clears the current path.
The \code{setlinewidth} command issued at the start of the code
block indicates that all subsequent stroke commands should
be drawn with a line width of $1/2$ of a point.  The default
line width in Proof by Picture is 1pt, so the resulting
circles are drawn thinner than the default.
After drawing the two circles, the diagram looks as follows.

\hfil\begin{pbpic}<8*cm,6*cm>
import pbpic.arrow
arrowto=arrow.ArrowTo();

W=7.;H=5.

scaleto(1*cm)
translate(0.5,0.5)

with gsave():
  moveto(0,0); build(paths.rect,w=W,h=H)
  clip()
  
  setlinewidth(2)
  setlinecolor(color.GrayColor(0.2))
  moveto(-0.5,-0.5); build(paths.vlines,h=H+1,M=int(W)+2)
  moveto(-0.5,-0.5); build(paths.hlines,w=W+1,N=int(H)+2)
  stroke()

moveto(0,0); build(paths.rect,w=7,h=5)
stroke()

translate(W/2,H/2)
R=1.5
A=point(-R/2,0)
B=point(R/2,0)
setlinewidth(0.5*pt)
for c in [A,B]:
  moveto(c);
  build(paths.circle,r=R)
  stroke()
\end{pbpic}

The path for the equilateral triangle can be built
using primitive path building commands.
\begin{snippet}
moveto(A); lineto(B); lineto(C); closepath()
stroke()
\end{snippet}
This code block moves the current point to $A$ with  \code{moveto}, 
and draws a straight lines to $B$ and $C$ with \code{lineto} commands.
The \code{closepath} command draws a straight back to $A$ and indicates 
that a closed loop has been created in the current path.  A more compact
and legible alternative is to build a polygon.
\begin{snippet}
build(paths.polygon,[A,B,C])
\end{snippet}

The triangle should be drawn with a heavier line than the circles,
since we want it to be the focus of the figure.  So we set the line width
to 1pt before issuing the \cmd{stroke} command.
\begin{snippet}
setlinewidth(1*pt)
build(paths.polygon,[A,B,C])
stroke()
\end{snippet}

\hfil\begin{pbpic}<8*cm,6*cm>
import pbpic.arrow
arrowto=arrow.ArrowTo();

W=7.;H=5.

scaleto(1*cm)
translate(0.5,0.5)

with gsave():
  moveto(0,0); build(paths.rect,w=W,h=H)
  clip()
  
  setlinewidth(2)
  setlinecolor(color.GrayColor(0.2))
  moveto(-0.5,-0.5); build(paths.vlines,h=H+1,M=int(W)+2)
  moveto(-0.5,-0.5); build(paths.hlines,w=W+1,N=int(H)+2)
  stroke()

moveto(0,0); build(paths.rect,w=7,h=5)
stroke()

translate(W/2,H/2)
R=1.5
A=point(-R/2,0)
B=point(R/2,0)
with ctmsave():
  translate(A)
  rotate(1./6)
  C=pagePoint(R,0)

setlinewidth(0.5*pt)
for c in [A,B]:
  moveto(c);
  build(paths.circle,r=R)
  stroke()
  
setlinewidth(1.*pt)
build(paths.polygon,[A,B,C])
stroke()
\end{pbpic}

The labels are added to the diagram using the \code{drawtex} command,
which is used to add text generated by LaTeX to the figure.
\begin{snippet}
moveto(A); rmoveto(-4*pt,0)
drawtex("$A$",origin=loc.east)
\end{snippet}
These command move the current point to $A$ and then shift
the currently point to the left by 4pt using \code{rmoveto} (relative
move to).  The \code{drawtex} command makes a little figure that contains
the letter $A$, and then places it so that its `easterly' boundary point is set down on top of the current point.  Adding the label at $B$ is
done similarly, with adjustments as needed to account for the mirror-image
directions.
\begin{snippet}
moveto(B); rmoveto(4*pt,0)
drawtex("$B$",origin=loc.west)
\end{snippet}

\hfil\begin{pbpic}<8*cm,6*cm>
import pbpic.arrow
arrowto=arrow.ArrowTo();

W=7.;H=5.

scaleto(1*cm)
translate(0.5,0.5)

with gsave():
  moveto(0,0); build(paths.rect,w=W,h=H)
  clip()
  
  setlinewidth(2)
  setlinecolor(color.GrayColor(0.2))
  moveto(-0.5,-0.5); build(paths.vlines,h=H+1,M=int(W)+2)
  moveto(-0.5,-0.5); build(paths.hlines,w=W+1,N=int(H)+2)
  stroke()

moveto(0,0); build(paths.rect,w=7,h=5)
stroke()

translate(W/2,H/2)
R=1.5
A=point(-R/2,0)
B=point(R/2,0)
with ctmsave():
  translate(A)
  rotate(1./6)
  C=pagePoint(R,0)

setlinewidth(0.5*pt)
for c in [A,B]:
  moveto(c);
  build(paths.circle,r=R)
  stroke()
  
setlinewidth(1.*pt)
build(paths.polygon,[A,B,C])
stroke()

moveto(A); rmoveto(-4*pt,0)
drawtex("$A$",origin=loc.east)

moveto(B); rmoveto(4*pt,0)
drawtex("$B$",origin=loc.west)
\end{pbpic}

The entire code for the figure is given in Listing \ref{lst:euclidI1}
\begin{snippet}
from pbpic import *

pbpbegin(7*cm,5*cm,'pdf')

scaleto(1*cm)
translate(loc.center)

R=1.5
A=point(-R/2,0)
B=point(R/2,0)
with ctmsave():
  translate(A)
  rotate(1./6)
  C=pagePoint(R,0)

setlinewidth(0.5*pt)
for c in [A,B]:
  moveto(c); build(paths.circle,r=R)
  stroke()

setlinewidth(1*pt)
moveto(A); lineto(B); lineto(C); closepath();
stroke()

moveto(A); rmoveto(-4*pt,0)
drawtex("$A$",origin=loc.east)

moveto(B); rmoveto(4*pt,0)
drawtex("$B$",origin=loc.west)

pbpend()
\end{snippet}

\subsection{Commentary}

This behaviour can be overridden by giving the full file name as the third argument to \code{pbpbegin}, in which case the extension of the 
file name is used
to determine the output format.

You
can determine the coordinates of \code{C} in the whatever the current
coordinate system happens to be using \code{point(C)}.

Angles in Proof by Picture
are usually specified in fractions of a circle.  A little care is needed
with factions in Python: by default the division of two integers results
in an integer, so \code{1/6} would be zero.  You can remedy this by
either writing \code{1./6} or issuing \code{from \_\_future\_\_ import division}
at the start of your document, which will result in the division of
any two integers yielding a floating point number. 

 
To describe the locations of the circles and triangles in the page, we'll 
need the three vertices of the triangle.  The points $A$ and $B$ are easy
to compute
The circles will have a radius
of 1.2cm, and the points $A$ and $B$ are the centers of these circles.
These points are easy to compute, 

When drawing at
the most elementary level, one builds a path and then 



The full program for this figure is given in Listing \ref{lst:ecuildI1}.

\begin{snippet}
from pbpic import *

pbpbegin(7*cm,5*cm,'pdf')

scaleto(1*cm)
translate(loc.center)

R=1.5
A=point(-R/2,0)
B=point(R/2,0)
with ctmsave():
  translate(A)
  rotate(1./6)
  C=pagePoint(R,0)

setlinewidth(0.5*pt)
for c in [A,B]:
  moveto(c); build(paths.circle,r=R)
  stroke()

setlinewidth(1*pt)
moveto(A); lineto(B); lineto(C); closepath();
stroke()

moveto(A); rmoveto(-4*pt,0)
drawtex("$A$",origin=loc.east)

moveto(B); rmoveto(4*pt,0)
drawtex("$B$",origin=loc.west)

pbpend()
\end{snippet}

\section{Pythagorean Theorem}

Book I Proposition 47 of Euclid's Elements proves the Pythagorean Theorem,
namely that the square of the hypotenuse of a right angle 
triangle equals the sum of the square of the other two sides.
The strategy of the proof is to show that the blue areas
and green areas in the figure below are the same

\hfil\begin{pbpic}<7*cm,7*cm>
from pbpic import *
import math

a=3; b=4; c=math.sqrt(a*a+b*b)

lightblue = color.RGBColor(0,.5,.9)
lightgreen = color.RGBColor(.1,.9,.5)

scaleto(.5*cm); translate(loc.center)

# Determine the locations of the triangle vertices
moveto(-a/2,-b/2); A = currentpoint()
rmoveto(a,0); B = currentpoint()
rmoveto(0,b); C = currentpoint()
newpath()

# Compute the projection of 'B' onto the line CA
V = (A-C).unitvector().perp()
D = B - (B-C).dot(V)*V

# Draw altitude of triangle
moveto(B); lineto(D);
stroke(color.GrayColor(.3))

# Draw triangle
with gsave():
  build(paths.polygon,[A,B,C])
  setlinejoin('bevel')
  stroke()

vertex = [A,B,C,D]
xlength = [a,b,(D-C).length(),(A-D).length()]
ylength = [a,b,c,c]
c = [lightgreen, lightblue, lightblue, lightgreen]

for i in range(4):
  P0=vertex[i]; P1=vertex[(i+1)%4]
  with ctmsave():
    translate(P0)
    rotate((P1-P0).angle())
    moveto(0,0);
    build(paths.rect,xlength[i],-ylength[i])
  with gsave():
    fill(c[i])
  stroke()

label = ['$A$', '$B$', '$C$' ]
(B-A).unitvector()+(C-A).unitvector()
vertex = [A,B,C]
for i in range(3):
  P=vertex[i]; Q=vertex[(i-1)%3]; R=vertex[(i+1)%3]
  moveto(P);
  S= -( (Q-P).unitvector() + (R-P).unitvector() )
  rmoveto(6*pt,S)
  drawtex(label[i],origin=loc.border(-S))

\end{pbpic}

We start this diagram starts by setting up a coordinate system: we'll
make a 7cm by 7cm figure, and choose an initial unit size of 1/2cm.

\begin{snippet}
from pbpic import *

pbpbegin(7*cm,7*cm,'pdf')
scaleto(.5*cm); translate(loc.center)

# Drawing commands go here.

pbpend()
\end{snippet}

The triangle itself is parameterized in terms of its two (non-hypotenuse)
side lengths $a=3$, $b=4$.  Every other point will be computed based
on these values; this way, we can adjust the diagram simply by 
changing $a$ and $b$.
\begin{snippet}
a=3; b=4; c=math.sqrt(a*a+b*b)

# Determine the locations of the triangle vertices
moveto(-a/2,-b/2); A = currentpoint()
rmoveto(a,0); B = currentpoint()
rmoveto(0,b); C = currentpoint()
newpath()

\end{snippet}

Rather than code the vertex locations directly (which would have been
a perfectly reasonable approach), we construct the the triangle vertices
via relative move-to operations, and grab the 
coordinates of each vertex as it is constructed.  The \code{currentpoint} command returns the coordinates of the path's current point in the current coordinate system.  
At the end of this computation, we call \code{newpath} to erase the
path and clean up our scratch work.

Having located $A$, $B$, and $C$, we also need the point $D$, which 
is the projection of $B$ onto the line $CA$.  If $V$ is a unit vector
perpendicular to $CA$, then
$$
D = B-[ (B-C)\cdot V] V.
$$

\hfil\begin{pbpic}<5*cm,5*cm>
arrowto=arrow.ArrowTo();

from pbpic import *
import math

a=3; b=4; c=math.sqrt(a*a+b*b)

lightblue = color.RGBColor(0,.5,.9)
lightgreen = color.RGBColor(.1,.9,.5)

scaleto(.5*cm); translate(loc.center)

# Determine the locations of the triangle vertices
moveto(-a/2,-b/2); A = currentpoint()
rmoveto(a,0); B = currentpoint()
rmoveto(0,b); C = currentpoint()
newpath()

# Compute the projection of 'B' onto the line CA
V = (A-C).unitvector().perp()
D = B - (B-C).dot(V)*V

# Draw triangle
with gsave():
  build(paths.polygon,[A,B,C])
  stroke()

V = (A-C).unitvector().perp()
tip = D+V
moveto(D); draw(arrowto,tip)

moveto(tip)
rmoveto(2*pt,2*pt)
drawtex('$V$',origin=loc.sw)

label = ['$A$', '$B$', '$C$' ]
(B-A).unitvector()+(C-A).unitvector()
vertex = [A,B,C]
for i in range(3):
  P=vertex[i]; Q=vertex[(i-1)%3]; R=vertex[(i+1)%3]
  moveto(P);
  S= -( (Q-P).unitvector() + (R-P).unitvector() )
  rmoveto(6*pt,S)
  drawtex(label[i],origin=loc.border(-S))

moveto(D)
with ctmsave():
  scaleto(2*pt)
  build(paths.circle,r=1)
  fill()

moveto(D)
rmoveto(6*pt,-V)
drawtex('$D$',origin=loc.border(V))

\end{pbpic}

The computation in Python of the point $D$ is done as follows.
\begin{snippet}
# Compute the projection of 'B' onto the line CA
V = (A-C).unitvector().perp()
D = B - (B-C).dot(V)*V
\end{snippet}
With all the important points computed, we now draw the diagram
in the following order:
\begin{enumerate}
  \item The gray altitude.
  \item The triangle.
  \item The four colored rectangles.
  \item The vertex labels.
\end{enumerate}

When we stroke the altitude, we can specify the color of the
line.
\begin{snippet}
# Draw altitude of triangle
moveto(B); lineto(D);
stroke(color.GrayColor(.3))
\end{snippet}
The \code{color.GrayColor} is a class that represents a grayscale
color in the range from 0 (white) to 1 (black), so a 
\code{color.GrayColor(.3)} is a light shade of gray.

The most straightforward way to draw the triangle is to 
use a polygon.
\begin{snippet}
build(paths.polygon,[A,B,C])
stroke()
\end{snippet}

However, the effect will have undesired side-effect.  Compare the two
figures below that have zoomed in on the point $C$.  The one on the
left shows that the triangle's tip has an unaesthetic extension.

\hfil\begin{pbpic}<4*cm,4*cm>
from pbpic import *
import math

a=3; b=4; c=math.sqrt(a*a+b*b)

lightblue = color.RGBColor(0,.5,.9)
lightgreen = color.RGBColor(.1,.9,.5)

with inset(4*cm,4*cm) as figure:
  scaleto(.5*cm); translate(loc.center)
  scale(2)
  setlinewidth(2*pt)

  # Determine the locations of the triangle vertices
  moveto(-a/2,-b/2); A = currentpoint()
  rmoveto(a,0); B = currentpoint()
  rmoveto(0,b); C = currentpoint()
  newpath()

  markpoint('C',C)

  # Compute the projection of 'B' onto the line CA
  V = (A-C).unitvector().perp()
  D = B - (B-C).dot(V)*V

  # Draw altitude of triangle
  moveto(B); lineto(D);
  stroke(color.GrayColor(.3))

  # Draw triangle
  with gsave():
    build(paths.polygon,[A,B,C])
    stroke()

  vertex = [A,B,C,D]
  xlength = [a,b,(D-C).length(),(A-D).length()]
  ylength = [a,b,c,c]
  c = [lightgreen, lightblue, lightblue, lightgreen]

  for i in range(4):
    P0=vertex[i]; P1=vertex[(i+1)%4]
    with ctmsave():
      translate(P0)
      rotate((P1-P0).angle())
      moveto(0,0);
      build(paths.rect,xlength[i],-ylength[i])
    with gsave():
      fill(c[i])
    stroke()

moveto(loc.center)
draw(figure,origin='C')

\end{pbpic}
\hfil\begin{pbpic}<4*cm,4*cm>
from pbpic import *
import math

a=3; b=4; c=math.sqrt(a*a+b*b)

lightblue = color.RGBColor(0,.5,.9)
lightgreen = color.RGBColor(.1,.9,.5)

with inset(4*cm,4*cm) as figure:
  scaleto(.5*cm); translate(loc.center)
  scale(2)
  setlinewidth(2*pt)

  # Determine the locations of the triangle vertices
  moveto(-a/2,-b/2); A = currentpoint()
  rmoveto(a,0); B = currentpoint()
  rmoveto(0,b); C = currentpoint()
  newpath()

  markpoint('C',C)

  # Compute the projection of 'B' onto the line CA
  V = (A-C).unitvector().perp()
  D = B - (B-C).dot(V)*V

  # Draw altitude of triangle
  moveto(B); lineto(D);
  stroke(color.GrayColor(.3))

  # Draw triangle
  with gsave():
    build(paths.polygon,[A,B,C])
    setlinejoin('bevel')
    stroke()

  vertex = [A,B,C,D]
  xlength = [a,b,(D-C).length(),(A-D).length()]
  ylength = [a,b,c,c]
  c = [lightgreen, lightblue, lightblue, lightgreen]

  for i in range(4):
    P0=vertex[i]; P1=vertex[(i+1)%4]
    with ctmsave():
      translate(P0)
      rotate((P1-P0).angle())
      moveto(0,0);
      build(paths.rect,xlength[i],-ylength[i])
    with gsave():
      fill(c[i])
    stroke()

moveto(loc.center)
draw(figure,origin='C')

\end{pbpic}

To correct this, we draw the triangle using:
\begin{snippet}
# Draw triangle
with gsave():
  build(paths.polygon,[A,B,C])
  setlinejoin('bevel')
  stroke()
\end{snippet}
The \code{setlinejoin} command determines how corners are treated 
when they are stroked.  There are three settings: \code{'miter'},
\code{'bevel'}, and \code{'round'} which look as follows.

\hfil\begin{pbpic}<9*cm,4*cm>
scaleto(1*cm);translate(1.5,2)
j=['miter','bevel','round']

setlinewidth(8*pt)

for i in range(3):
  moveto(.5,-.75)
  rlineto(0,1.5)
  rlineto(-1,-1.5)
  setlinejoin(j[i])
  stroke()
  moveto(0,-1.5)
  rmoveto(0,-4*pt)
  drawtex(j[i],origin=loc.north)
  
  translate(3,0)
\end{pbpic}

The line join style is one of a number of parameters that comprise
the graphics state, which is maintained throughout drawing the diagram.
Because we only want to affect the line join style for the triangle, and
not for the rest of the figure, we use the \code{with gsave():} command.
On exiting the \code{gsave} code block, the graphics state will be reset to 
whatever it was when we entered it.  So by wrapping our triangle drawing
with a \code{gsave}, we ensure that we only temporarily change the
line join style.  

At this stage, our diagram looks as follows.

\hfil\begin{pbpic}<7*cm,7*cm>
from pbpic import *
import math

a=3; b=4; c=math.sqrt(a*a+b*b)

lightblue = color.RGBColor(0,.5,.9)
lightgreen = color.RGBColor(.1,.9,.5)

scaleto(.5*cm); translate(loc.center)

# Determine the locations of the triangle vertices
moveto(-a/2,-b/2); A = currentpoint()
rmoveto(a,0); B = currentpoint()
rmoveto(0,b); C = currentpoint()
newpath()

# Compute the projection of 'B' onto the line CA
V = (A-C).unitvector().perp()
D = B - (B-C).dot(V)*V

# Draw altitude of triangle
moveto(B); lineto(D);
stroke(color.GrayColor(.3))

# Draw triangle
with gsave():
  build(paths.polygon,[A,B,C])
  setlinejoin('bevel')
  stroke()
\end{pbpic}

We now draw the rectangles on the triangle sides.  The following
sample code shows how to draw a single rectangle, filled with
the color red, and stroked around the outside.
\begin{snippet}
dx=2; dy=1.5
moveto(-dx/2,-dy/2)
build(paths.rect,dx,dy)
with gsave():
  fill(color.red)
stroke()
\end{snippet}
\hfil\begin{pbpic}<4*cm,4*cm>
scaleto(1*cm); translate(loc.center)
dx=2; dy=1.5
moveto(-dx/2,-dy/2)
build(paths.rect,dx,dy)
with gsave():
  fill(color.red)
stroke()
moveto(0,-dy/2)
rmoveto(0,-4*pt)
drawtex('$dx$',origin=loc.north)

moveto(-dx/2,0)
rmoveto(-4*pt,0)
drawtex('$dy$',origin=loc.east)

moveto(-dx/2,-dy/2)
rmoveto(-4*pt,-4*pt)
drawtex('$O$',origin=loc.ne)
\end{pbpic}

Note the use of \code{gsave} here as well.  The current path
is part of the graphics state.  Issuing a \code{fill} or \code{stroke}
command clears the current path.  Since we want to both fill and stroke
the rectangle, we save the graphics state, fill the path, restore
the graphics state (which restores the path) and finally stroke it.
This combination of commands is sufficiently common that there is a shortcut:
\begin{snippet}
fillstroke(fillcolor=color.red)
\end{snippet} 
does the same thing as
\begin{snippet}
with gsave():
  fill(color.red)
stroke()
\end{snippet}

To draw all four rectangles, we set up the parameters for
each of them in advance, and then use a loop.  For each rectangle,
we need to know its lower left corner, and its two side lengths.
These are the variables \code{vertex}, \code{xlength}, and \code{ylength}
in the code.
\begin{snippet}
vertex = [A,B,C,D]
xlength = [a,b,(D-C).length(),(A-D).length()]
ylength = [a,b,c,c]

lightblue = color.RGBColor(0,.5,.9)
lightgreen = color.RGBColor(.1,.9,.5)
c = [lightgreen, lightblue, lightblue, lightgreen]
\end{snippet}
The variable \code{c} contains the rectangle colors, which are
specified using \code{color.RGBColor} objects that specify the
red, green, and blue components of a color, each in the range from 0 to 1.
For example, \code{color.red} is equivalent to \code{color.RGBColor(1,0,0)}.

To draw each rectangle, we translate the coordinate system
to its origin, then rotate the coordinates to align the rectangle base with the triangle side.  In the code, the variables $P0$ and $P1$ contain 
the two vertices
of the rectangle on the triangle side, and \code{(P1-P0).angle()} computes 
the angle between the vector vector \code{P1-P0} and the $x$-axis.  This
is the correct angle to rotate by.  All these coordinate transformations
are wrapped with a \code{with ctmsave():} so that they are easily undone.
\begin{snippet}
for i in range(4):
  P0=vertex[i]; P1=vertex[(i+1)%4]
  with ctmsave():
    translate(P0)
    rotate((P1-P0).angle())
    moveto(0,0);
    build(paths.rect,xlength[i],-ylength[i])
  fillstroke(fillcolor=c[i])
\end{snippet}

\hfil\begin{pbpic}<7*cm,7*cm>
from pbpic import *
import math

a=3; b=4; c=math.sqrt(a*a+b*b)

lightblue = color.RGBColor(0,.5,.9)
lightgreen = color.RGBColor(.1,.9,.5)

scaleto(.5*cm); translate(loc.center)

# Determine the locations of the triangle vertices
moveto(-a/2,-b/2); A = currentpoint()
rmoveto(a,0); B = currentpoint()
rmoveto(0,b); C = currentpoint()
newpath()

# Compute the projection of 'B' onto the line CA
V = (A-C).unitvector().perp()
D = B - (B-C).dot(V)*V

# Draw altitude of triangle
moveto(B); lineto(D);
stroke(color.GrayColor(.3))

# Draw triangle
with gsave():
  build(paths.polygon,[A,B,C])
  setlinejoin('bevel')
  stroke()

vertex = [A,B,C,D]
xlength = [a,b,(D-C).length(),(A-D).length()]
ylength = [a,b,c,c]
c = [lightgreen, lightblue, lightblue, lightgreen]

for i in range(4):
  P0=vertex[i]; P1=vertex[(i+1)%4]
  with ctmsave():
    translate(P0)
    rotate((P1-P0).angle())
    moveto(0,0);
    build(paths.rect,xlength[i],-ylength[i])
  fillstroke(fillcolor=c[i])
\end{pbpic}

To add the vertex labels, we again use a loop.  The
hard part is positioning the labels automatically.
For each vertex, we compute the vector \code{S} that
points outside the triangle along the line that bisects
the angle at that vertex.  The command \code{rmoveto(6*pt,S)}
moves the current point 6pt in the direction of the vector $S$,
and the command \code{drawtex(label[i],origin=loc.border(-S))}
positions the label so THIS NEEDS A PICTURE.
\begin{snippet}
label = ['$A$', '$B$', '$C$' ]
vertex = [A,B,C]
for i in range(3):
  P=vertex[i]; Q=vertex[(i-1)%3]; R=vertex[(i+1)%3]
  moveto(P);
  S= -( (Q-P).unitvector() + (R-P).unitvector() )
  rmoveto(6*pt,S)
  drawtex(label[i],origin=loc.border(-S))
\end{snippet}

The total code for the figure follows.
\begin{snippet}
from pbpic import *
import math

a=3; b=4; c=math.sqrt(a*a+b*b)

lightblue = color.RGBColor(0,.5,.9)
lightgreen = color.RGBColor(.1,.9,.5)

pbpbegin(7*cm,7*cm,'pdf')
scaleto(.5*cm); translate(loc.center)

# Determine the locations of the triangle vertices
moveto(-a/2,-b/2); A = currentpoint()
rmoveto(a,0); B = currentpoint()
rmoveto(0,b); C = currentpoint()
newpath()

# Compute the projection of 'B' onto the line CA
V = (A-C).unitvector().perp()
D = B - (B-C).dot(V)*V

# Draw altitude of triangle
moveto(B); lineto(D);
stroke(color.GrayColor(.3))

# Draw triangle
with gsave():
  build(paths.polygon,[A,B,C])
  setlinejoin('bevel')
  stroke()

# Draw rectangles
vertex = [A,B,C,D]
xlength = [a,b,(D-C).length(),(A-D).length()]
ylength = [a,b,c,c]
c = [lightgreen, lightblue, lightblue, lightgreen]

for i in range(4):
  P0=vertex[i]; P1=vertex[(i+1)%4]
  with ctmsave():
    translate(P0)
    rotate((P1-P0).angle())
    moveto(0,0);
    build(paths.rect,xlength[i],-ylength[i])
  with gsave():
    fill(c[i])
  stroke()

# Draw labels
label = ['$A$', '$B$', '$C$' ]
(B-A).unitvector()+(C-A).unitvector()
vertex = [A,B,C]
for i in range(3):
  P=vertex[i]; Q=vertex[(i-1)%3]; R=vertex[(i+1)%3]
  moveto(P);
  S= -( (Q-P).unitvector() + (R-P).unitvector() )
  rmoveto(6*pt,S)
  drawtex(label[i],origin=loc.border(-S))
pbpend()
\end{snippet}

\end{document}
