%& -shell-escape
\documentclass[12pt]{article}
\usepackage{pbpic,graphicx}
\usepackage{amsmath,amsfonts}
\usepackage{MinionPro}
\usepackage{fancyvrb}
\def\prg#1{{\tt #1}}
\def\pbp{Proof by Picture }
\def\Python{python}
\def\cmd#1{{\tt #1}}

\usepackage{listings}
\lstnewenvironment{snippet}{\lstset{language=Python,xleftmargin=12pt,xrightmargin=12pt,basicstyle=\small\ttfamily,escapechar=|}}{}

\begin{document}
\section{Introduction}
\pbp{} is a package for making diagrams, particularly
mathematical diagrams.  The package is built on top of the
\Python{} programming language; a picture is described via
a sequence of python commands.  The world's smallest
\pbp program is the following.
\begin{snippet}
from pbpic import *
pbpbegin(2*inch,2*inch,'pdf')
pbpend()
\end{snippet}
It would be worthwhile at this stage to actually run this code.
Save it to a file, e.g. \verb|tiny.py| and then run the file
as a python script (e.g. execute \verb|python tiny.py| at the command line).
The result should be a file \verb|tiny.pdf| that contains an empty 2 inch
by 2 inch figure.

There are two basic commands for making marks on the page:
\cmd{stroke} and \cmd{fill}.  Before these commands are issued,
a path must be constructed that contains the outline of the
region to be drawn.  For example:
\begin{snippet}
pbpbegin(1*inch,1*inch,'pdf')
scaleto(1*inch);
moveto(0,0); lineto(1,1);
moveto(0,1); lineto(1,0);
stroke()
pbpend()
\end{snippet}
\hfil\begin{pbpic}<1*inch,1*inch>
scaleto(1*inch);
moveto(0,0); lineto(1,1);
moveto(0,1); lineto(1,0);
stroke()  
\end{pbpic}

The 

% \vskip 12pt
% 
% \def\eg0{path() + (0,0)-(40,40);stroke()}
% \pbpicexec<2*inch,2*inch>{\eg0}
% 
% \begin{snippet}
% from pbpic import *
% |\eg0|
% pbpend()
% \end{snippet}
% 
% \hfil
% \begin{pbpic}<2*inch,2*inch>
% scaleto(1*inch)
% translate(loc.center)
% moveto(-0.5,-0.5); rlineto(1,1)
% moveto(-0.5,0.5); rlineto(1,-1)
% stroke()
% \end{pbpic}

% 
% \begin{python}
% from pbpic import *
% pbpbegin(2*inch,2*inch,'tiny.pdf')
% pbpend()
% \end{python}
% \includegraphics{tiny}


\section{Coordinate Systems}

A \pbp diagram is described using geometric objects, such
as points and vectors, that are tracked
internally in a system of coordinates on $\mathbb{R}^2$ known 
as {\bf page coordinates}.  On the other hand, 
a \pbp program need not describe the position of these
objects using page coordinates; instead, it indicates
the locations of these objects using {\bf user coordinates},
which are always related to page coordinates via an affine
transformation $T$.  Thus, the command \prg{moveto(3,2)}
moves the current point to the point $(a,b)$ in page
coordinates, where
$$
(a,b) = T(3,2).
$$


Lengths are ovals.  They are used to convert a vector in page
coordinates into a rescaled vector with the length given in the Length's
coordinate system

\end{document}